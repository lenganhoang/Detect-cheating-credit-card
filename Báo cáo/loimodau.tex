\chapter*{\centerline{\bf \large\MakeUppercase{Lời mở đầu}}}
\vspace{20pt}

Ngày nay, thẻ tín dụng đang dần trở thành một trong những phương tiện giao dịch phổ biến. Cũng chính nhờ vậy, các trường hợp gian lận thẻ tín dụng xảy ra ngày càng nhiều, chúng ta cần có phương pháp phát hiện, ngăn chặn để bảo vệ khách hàng. Vì vậy, bài toán đặt ra rằng cần có một phương pháp để đánh giá, xếp loại, phát hiện các loại gian lận thẻ tín dụng thông qua các giao dịch của khách hàng. Ở đây, cụ thể nhóm tập trung nghiên cứu phương pháp sử dụng mô hình Markov ẩn để phát hiện gian lận thẻ tín dụng.

Dưới đây nhóm sẽ trình bày bài toán, phương pháp giải, các thuật toán và chương trình để giải bài toán này. 

\begin{itemize}
\item Chương 1: Trình bày nội dung và những thông tin cơ bản của bài toán
\item Chương 2: Giới thiệu về mô hình Markov ẩn và cách áp dụng mô hình Markov ẩn vào bài toán phát hiện gian lận thẻ tín dụng.
\item Chương 3: Trình bày về về kết quả của bài toán với dữ liệu giao dịch
\end{itemize}
Do thời gian thực hiện báo cáo không nhiều, kiến thức còn hạn chế nên khi làm báo cáo không tránh khỏi những sai sót. Nhóm mong nhận được sự góp ý và những ý kiến phản biện của quý thầy cô và bạn đọc.


\textrm{Nhóm xin chân thành cảm ơn!}
  \begin{flushright}
{\it Hà Nội, ngày 10 tháng 06 năm 2019}

 \end{flushright}



